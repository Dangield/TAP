% TeX encoding = utf8
% TeX spellcheck = pl_PL 
\documentclass[a4paper,titlepage,11pt,twosides,floatssmall]{mwrep}
\usepackage[left=2.5cm,right=2.5cm,top=2.5cm,bottom=2.5cm]{geometry}
\usepackage[OT1]{fontenc}
\usepackage{polski}
\usepackage{amsmath}
\usepackage{amsfonts}
\usepackage{amssymb}
\usepackage{graphicx}
\usepackage{url}
\usepackage{tikz}
\usetikzlibrary{arrows,calc,decorations.markings,math,arrows.meta}
\usepackage{rotating}
\usepackage[percent]{overpic}
\usepackage[utf8]{inputenc}
\usepackage{xcolor}
\usepackage{pgfplots}
\usetikzlibrary{pgfplots.groupplots}
\usepackage{listings}
\usepackage{matlab-prettifier}
\usepackage{siunitx}
\usepackage[section]{placeins}
\definecolor{szary}{rgb}{0.95,0.95,0.95}
\SendSettingsToPgf
\sisetup{detect-weight,exponent-product=\cdot,output-decimal-marker={,},per-mode=symbol,binary-units=true,range-phrase={-},range-units=single}

%konfiguracje pakietu listings
\lstset{
	backgroundcolor=\color{szary},
	frame=single,
	breaklines=true,
}
\lstdefinestyle{customlatex}{
	basicstyle=\footnotesize\ttfamily,
	%basicstyle=\small\ttfamily,
}
\lstdefinestyle{customc}{
	breaklines=true,
	frame=tb,
	language=C,
	xleftmargin=0pt,
	showstringspaces=false,
	basicstyle=\small\ttfamily,
	keywordstyle=\bfseries\color{green!40!black},
	commentstyle=\itshape\color{purple!40!black},
	identifierstyle=\color{blue},
	stringstyle=\color{orange},
}
\lstdefinestyle{custommatlab}{
	captionpos=t,
	breaklines=true,
	frame=tb,
	xleftmargin=0pt,
	language=matlab,
	showstringspaces=false,
	%basicstyle=\footnotesize\ttfamily,
	basicstyle=\scriptsize\ttfamily,
	keywordstyle=\bfseries\color{green!40!black},
	commentstyle=\itshape\color{purple!40!black},
	identifierstyle=\color{blue},
	stringstyle=\color{orange},
}

%wymiar tekstu (bez żywej paginy)
\textwidth 160mm \textheight 247mm

%ustawienia pakietu pgfplots
\pgfplotsset{
	tick label style={font=\scriptsize},
	label style={font=\small},
	legend style={font=\small},
	title style={font=\small}
}

\def\figurename{Rys.}
\def\tablename{Tab.}

%konfiguracja liczby pływających elementów
\setcounter{topnumber}{0}%2
\setcounter{bottomnumber}{3}%1
\setcounter{totalnumber}{5}%3
\renewcommand{\textfraction}{0.01}%0.2
\renewcommand{\topfraction}{0.95}%0.7
\renewcommand{\bottomfraction}{0.95}%0.3
\renewcommand{\floatpagefraction}{0.35}%0.5

\begin{document}
	
	\begin{titlepage}
		\begin{center}
			\Huge{\textsc{Sprawozdanie z pierwszej części projektu z przedmiotu \\,,Technika Automatyzacji Procesów''}} \\
			[15cm]
			\Large{Numer zadania: 3 \\Wykonawcy:}\\
			\Large{Dawidiuk Marek \\ Giełdowski Daniel \\ Kłos Maciej \\ Taras Sylwia}
		\end{center}
	\end{titlepage}
	
	\tableofcontents
	\newpage
	\chapter{Opis otrzymanego modelu}

W ramach tego projektu korzystaliśmy z modelu obiektu znanego jako reaktor przepływowy. Obiekt składa się z pojemnika wypełnionego cieczą z rozpuszczoną nieokreśloną substancją. Do pojemnika wpływa strumieniem $F_{in}$ ciecz o określonej temperaturze $T_{in}$ oraz stężeniu rozpuszczonej substancji $C_{Ain}$. W pojemniku jest określona ilość cieczy $V$ w określonej temperaturze $T$. Ciecz z pojemnika wypływa strumieniem $F$, zawierając stężenie $C_A$ rozpuszczonej substancji. Dodatkowo przez pojemnik przeprowadzona jest rura odpowiedzialna za chłodzenie bądź podgrzewanie, którą strumieniem $F_C$ płynie ciecz o temperaturze wejściowej $T_{Cin}$. Obiekt opisany jest następującymi równaniami:
\begin{equation}
	\left\{
	\begin{tabular}{l}
	$V \cdot \frac{dC_A}{dt}=F_{in} \cdot C_{Ain}-F \cdot C_A-V \cdot k_0 \cdot e^{-\frac{E}{R \cdot T}} \cdot C_A$\\
	$V \cdot \rho \cdot c_p \cdot \frac{dT}{dt}=F_{in} \cdot \rho \cdot c_p \cdot T_{in}-F \cdot \rho \cdot c_p \cdot T+V \cdot h \cdot k_0 \cdot e^{-\frac{E}{R \cdot T}} \cdot C_A - \frac{a \cdot (F_C)^{b+1}}{F_C+\frac{a \cdot (F_C)^b}{2 \cdot \rho_c \cdot c_{pc}}} \cdot (T-T_{Cin})$
	\end{tabular}
	\right.
\end{equation}
W równaniach występują stałe o podanych wartościach:\\
\begin{itemize}
	\item $\rho=\rho_c=10^6\frac{g}{m^3}$
	\item $c_p=c_{pc} = 1 \frac{cal}{g\cdot K}$
	\item $k_0 = 10^{10} \frac{1}{min}$
	\item $\frac{E}{R} = 8330,1 \frac{1}{K}$
	\item $h = 130\cdot 10^6 \frac{cal}{kmol}$
	\item $a = 1,678\cdot 10^6\frac{cal}{K\cdot m^3}$
	\item $b = 0,5$
\end{itemize}
Otrzymaliśmy także dane odnośnie wartości zmiennych modelu w zadanym punkcie pracy układu:\\
\begin{itemize}
	\item $V=1m^3$
	\item $F_{in} = F = 1 \frac{m^3}{min}$
	\item $C_{Ain} = 2 \frac{kmol}{m^3}$
	\item $F_C = 15 \frac{m^3}{min}$
	\item $T_{in} = 323K$
	\item $T_{Cin} = 365K$
	\item $C_A = 0,26\frac{kmol}{m^3}$
	\item $T = 393,9K$
\end{itemize}
W ramach zadania zmienne te podzielone zostały na 4 grupy:\\
\begin{itemize}
	\item stałe - $V,F,F_{in}$
	\item regulowane - $C_A,T$
	\item sterujące - $C_{Ain},F_C$
	\item zakłócenia - $T_{in},T_{Cin}$
\end{itemize}
Po zastąpieniu stałych w równaniach liczbami, otrzymywany jest następujący układ równań:\\
\begin{equation}
	\left\{
	\begin{tabular}{l}
	$\frac{dC_A}{dt} = C_{Ain} - C_A - 10^{10}\cdot e^{-\frac{8330,1}{T}}\cdot C_A$\\
	$\frac{dT}{dt} = T_{in} - T + 130\cdot 10^{10}\cdot e^{-\frac{8330,1}{T}}\cdot C_A-\frac{1,678\cdot (F_C)^{1,5}}{F_C+0,839\cdot (F_C)^{0,5}}(T-T_{Cin})$
	\end{tabular}
	\right.
\end{equation}
Ostatnią rzeczą jaka musiała zostać wzięta pod uwagę było dokładniejsze określenie wartości wyjść w punkcie pracy, ponieważ te podane w zadaniu były przybliżone. Proces ten został wykonany w pierwszej części projektu, a wartości wyjść wyniosły w przybliżeniu $C_A = 0,2646$ oraz $T = 393,9531$.

	
\end{document}